\documentclass[../main.tex]{subfiles}


\begin{document}
    \textbf{ARP (Address Resolution Protocol)} stosowany jest w sieciach Ethernet (jeśli w warstwie sieci wykorzystywany jest protokół
    IPv4), był też używany w sieciach Token Ring. W wersji IPv6 protokół ARP nie jest w ogóle wykorzystywany, zastępują go inne mechanizmy.\\

    Struktura ramki ARP:
    \begin{itemize}
        \item Typ sprzętu (2 oktety)
        \item Typ protokołu (2 oktety)
        \item Długość adresu sprzętu (1 oktet)
        \item Długość adresu protokołu (1 oktet)
        \item Kod operacji (2 oktety)
        \item Adres sprzętu nadawcy (dla Ethernet 6 oktetów)
        \item Adres protokołu nadawcy (dla IPv4 4 oktety)
    \end{itemize}

    \subsubsection{Wykrywanie zduplikowanych adresów IP}
    Tzw "zbędny ARP"
    \begin{itemize}
        \item Węzeł wysyła ARP Request z zapytaniem o swój własny adres.
        \begin{itemize}
            \item Jeśli ARP Reply nie nadejdzie to znaczy, że w lokalnym segmencie nie ma konfliktu adresów.
            \item Jeśli odpowiedź nadejdzie, oznacza to konflikt.
        \end{itemize}
        \item Węzeł już skonfigurowany traktowany jest jako węzeł z poprawnym adresem (\textbf{węzeł zgodny}, defending node), węzeł
        wysyłający „zbędny ARP” jest \textbf{węzłem konfliktowym} (offending node).
        \item \textbf{Węzeł konfliktowy wprowadza błąd} w pamięci podręcznej ARP komputerów w \textbf{całym segmencie} sieci. ARP Reply z węzła zgodnego nie naprawia sytuacji (ramka ARP Reply nie jest ramką rozgłoszeniową), więc zgodny wysyła ARP Request ze swoim adresem po wykryciu konfliktu.
    \end{itemize}
    Datagramy IP wysłane na w ramkach z niepoprawnym adresem MAC odbiorcy
    przepadają. Protokół IP nie zapewnia niezawodnej dostawy datagramów i nie
    spowoduje powtórnego przesłania datagramu w nowej ramce. Za niezawodność
    odpowiedzialne są protokoły warstwy transportu.

    \subsubsection{Proxy ARP}
    Router ze skonfigurowanym mechanizmem Proxy ARP odpowiada na ramki ARP Request w
    imieniu wszystkich węzłów – komputerów spoza segmentu sieci lokalnej. Może
    być używany jest np. w sytuacji, gdy komputery w sieci nie mają ustawionego domyślnego
    routera (domyślna brama, default gateway). Routery mogą mieć włączoną standardowo
    opcję Proxy ARP, wówczas jeśli jakiś komputer wyśle ARP Request z adresem spoza danej
    sieci lokalnej (zwykle to nie następuje), to router odpowie „w imieniu” komputera
    zewnętrznego.


    \subsubsection{Komunikacja między komputerami}

    Założenia:
    \begin{itemize}
        \item Komputer źródłowy - Komputer 1: IP1, MAC1
        \item Komputer docelowy - Komputer 2: IP2, MAC2
    \end{itemize}
    Połączone switchem. Na komputerze docelowym jest serwer strony WWW2.

    Jeżeli na komputerze 1 ktoś spróbuje otworzyć WWW2, to:

    \begin{itemize}
        \item Zadziała system DNS: komputer 1 skontaktuje się ze swoim serwerem DNS i zapyta jaki jest adres IP komputera związanego z nazwą domenową WW2. Serwer DNS znajdzie odpowiedni adres w swoich zasobach i odeśle informację do komputera 1.
        \item Przeglądarka utworzy komunikat (wg protokołu HTTP). Do komunikatu zostanie dodany nagłówek    (wg protokołu TCP), który zawiera m.in. port docelowy (standardowo 80)    oraz port źródłowy. Komunikat razem z dołączonym nagłówkiem TCP nazywa się    \textbf{segmentem TCP}.
        \item Do segmentu TCP zostanie dodany nagłówek IP – w ten sposób powstanie \textbf{pakiet IP}.
        \item Pakiet musi być przesłany w ramce. Do pakietu musi zostać dodany nagłówek ramki, zawierający źródłowy i docelowy adres MAC. \textbf{Komputer 1 nie zna adresu MAC komputera 2}. Zna tylko jego adres IP. Wykorzystywany jest \textbf{protokół ARP} – Address Resolution Protocol.
        \begin{itemize}
            \item Komputer 1 wysyła specjalną ramkę    \textbf{ARP Request} (ta NIE zawiera pakietu IP), która ma adres rozgłoszeniowy jako adres docelowy (same jedynki).
            \item Każdy komputer przyłączony do przełącznika ma obowiązek odebrać ramkę wysłaną na adres rozgłoszeniowy MAC. Jednak tylko komputer o zadanym IP odpowie na ARP Request.
            \item Odpowiedź to specjalna ramka \textbf{ARP Reply}. Odpowiedź ARP jest wysyłana na adres MAC komputera 1.
        \end{itemize}
        \item Po tym, jak komputer 1 pozna adres MAC komputera 2, może już zbudować ramkę przeznaczoną do komputera 2. Ramka jest wysyłana do przełącznika, a przełącznik dostarcza ją tylko do komputera 2.
        \begin{itemize}
            \item Przełącznik uczy się adresów MAC przyłączonych komputerów i routerów i zapamiętuje w tablicy przypisanie adresu MAC do konkretnego swojego portu.
        \end{itemize}
        \item Komputer 2 odbiera ramkę, sprawdza adres MAC docelowy i sumę kontrolną, po czym „wyjmuje” z ramki pakiet IP. Sprawdza adres docelowy IP i „wyjmuje” z pakietu segment TCP. Sprawdza do którego portu należy przekazać zawartość (komunikat HTTP) i ostatecznie „wyjmuje” komunikat http z segmentu i przekazuje do portu 80, na którym nasłuchuje serwer WWW.
        \item Serwer WWW konstruuje odpowiedź – stronę WWW. Strona ta zostanie umieszczona w komunikacie http, który następnie musi być przesłany do komputera 1. Mechanizm jest analogiczny jak poprzednio.
    \end{itemize}
    W rzeczywistości zanim może zostać przesłany segment TCP, komputery wykorzystujące ten protokół do komunikacji, muszą zbudować tzw. połączenie TCP.


    \begin{tabular}{|c|c|c|c|c|}
        \hline
        Nagłówek ramki & Nagłówek IP & Nagłówek TCP & Komunikat HTTP & Suma kontrolna\\
        (numery MAC) & (numery IP) & (numery portów) & & \\
        & 20 bajtów & 20 bajtów & & 4 bajty\\
        \hline
    \end{tabular}

    \textbf{W przypadku komunikacji między komputerami rozdzielonymi przynajmniej jednym routerem}
    \begin{itemize}
        \item Wszystko do skonstruowania pakietu IP włącznie działa tak samo. Komputer tworzący ramkę musi więc wykorzystując ARP Request poznać
        MAC adres routera, czyli swojej \textbf{bramy domyślnej}.
        \item Ramka jest wysyłana do routera.
        \item Router (brama) po otrzymaniu ramki „wyjmuje” z niej pakiet IP, zagląda do nagłówka i sprawdza jaki jest adres docelowy IP. Na podstawie
        tego adresu i tablicy routowania wyznacza router następnego skoku i konstruuje i wysyła do niego nową ramkę, w której umieszcza przesyłany pakiet IP. Analogicznie aż pakiet dotrze w kolejnych ramkach do docelowej sieci i do docelowego komputera.
    \end{itemize}


\end{document}