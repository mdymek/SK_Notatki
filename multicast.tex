\documentclass[../main.tex]{subfiles}
\begin{document}

    Multicast – transmisja grupowa, multiemisja.
    \begin{itemize}
        \item Wysłanie jednego pakietu ze źródła do wielu miejsc docelowych. Pakiety są kopiowane
        w routerach i przełącznikach warstwy drugiej.
        \item mniejsze obciążenie
        sieci, większa skalowalność w stosunku do unicastu
        \item Wykorzystanie: programy radiowe i telewizyjne, wideokonferencje, zdalne nauczanie,
        dystrybucja plików, ogłoszenia, monitoring, gry itd.
        \item schematy jeden-do-wielu, wiele-do-wielu.
        \item Komunikaty w większości protokołów routowania mają zarezerwowane adresy multiemisji.
        \begin{itemize}
            \item 224.0.0.1 – wszystkie komputery uczestniczące w transmisji grupowej (również routery) w
            segmencie sieci lokalnej.
            \item 224.0.0.2 – wszystkie routery uczestniczące w transmisji grupowej (multicast routers) w
            segmencie sieci lokalnej.
            \item 224.0.0.4 – routery DVMRP.
            \item 224.0.0.5 – wszystkie routery OSPF.
            \item 224.0.0.6 – routery DR OSPF.
            \item 224.0.0.9 – routery RIPv2 (RIPv1 wykorzystuje rozgłoszenie – broadcast, nie multicast).
            \item 224.0.0.0 – 239.255.255.255 - klasa D adresów IPv4
        \end{itemize}
        \item Aby uczestniczyć w transmisji grupowej, komputer musi sprawdzać określone adresy w
        przychodzących pakietach (IP) i generalnie w ramkach (MAC).
        \item Transmisja grupowa odbywa się z wykorzystaniem różnych mechanizmów i protokołów.
        \begin{itemize}
            \item Do komunikacji router – host wykorzystywany jest specjalny protokół IGMP.
            \item Ponadto routery wykorzystuj
            \begin{itemize}
                \item DVMRP – Distance Vector Multicast Routing Protocol,
                \item MOSPF – Multicast OSPF,
                \item PIM DM – Protocol Independent Multicast Dense Mode,
                \item PIM SM Sparse Mode, PIM Sparse-Dense Mode.
            \end{itemize}
        \end{itemize}
    \end{itemize}


    \subsection{IGMP - Internet Group Management Protocol}
    \begin{itemize}
        \item wykorzystywany do dynamicznego rejestrowania/wyrejestrowania
        odbiornika w routerze
        \item komunikaty IGMP są przesyłane w pakietach IP z adresem docelowym typu multicast i
        ustawioną wartością TTL na 1.
    \end{itemize}


    \subsubsection{IGMPv1}

    Są dwa typy komunikatów:
    \begin{itemize}
        \item Membership query (general membership query), wysyłany jest okresowo (co
        kilkadziesiąt sekund) przez routery na adres 224.0.0.1 (adres ten oznacza wszystkie
        komputery wykorzystujące multicast); służy do sprawdzenia, czy w sieci (na łączu) są
        odbiorcy dla emisji grupowej.
        \item Membership report, wysyłany jest na pewien adres grupy (na przykład 232.32.32.32),
        służy do zgłoszenia się jako odbiorca pakietów wysyłanych na ten adres; membership
        report wysyłany jest też w odpowiedzi na membership query.
    \end{itemize}


    Host (nie router) po otrzymaniu membership query czeka pewien pseudolosowy czas (z
    zakresu od 0 do 10 sekund) i wysyła membership report. Jeśli w tym pseudolosowym czasie
    host usłyszy membership report od innego hosta, to nie wysyła swojego raportu.
    W IGMPv1 host „po cichu” opuszcza grupę. Jeśli router nie dostanie raportu w odpowiedzi
    na trzy kolejne membership query (lub według innej reguły opisanej w odpowiednim RFC),
    router usuwa grupę z tablicy multicastu i przestaje przesyłać pakiety kierowane do tej grupy.
    W IGMPv1 nie ma mechanizmu wyboru jednego routera odpytującego (tzw. querier) w
    jednym segmencie sieci wykorzystującej technologię wielodostępu z wieloma routerami
    (multiple access, np. Ethernet albo Frame Relay). Mechanizm ten wprowadzono w IGRP v2.
    Podobny mechanizm jest też wykorzystywany w protokole routowania multicastu o nazwie
    PIM, routerem tym zostaje tzw. PIM Designated Router,
    czyli ten, którego adres IPv4 jest największą liczbą (32 bity).

    \subsubsection{IGMPv2}

    W IGMPv2 są cztery typy komunikatów:
    \begin{itemize}
        \item Membership query
        \item Version 1 membership report
        \item Version 2 membership report
        \item Leave group
    \end{itemize}


    Ważne zmiany w porównaniu do wersji pierwszej:
    \begin{itemize}
        \item Membership query może być typu group-specific query – zapytanie o członkostwo
        jest wysyłane na adres grupy zamiast na 224.0.0.1. W ten sposób router może
        sprawdzić, czy jest uczestnik w konkretnej grupie, bez proszenia uczestników
        wszystkich grup o raporty.
        \item Leave group message – komunikat o opuszczeniu grupy, wysyłany jest na adres
        224.0.0.2 (wszystkie routery multicast na łączu). Powoduje szybsze usunięcie grupy z
        tablicy multicastu, jeśli nie ma w niej odbiorców. Standardowo router po otrzymaniu
        tego komunikatu, wysyła group-specific query, żeby upewnić się, czy jest jeszcze jakiś
        uczestnik tej grupy na łączu. Cały proces opuszczenia i usunięcia grupy trwa typowo
        1-3 sekundy.
        \item Dodano do zapytań IGMP określenie czasu query-interval response time
        (max. resp. time), jaki mają uczestnicy na wysłanie raportu, czas ten jest określany przez
        wysyłającego zapytanie. Host po otrzymaniu membership query ustawia liczniki
        opóźnień dla każdej grupy (z wyjątkiem 224.0.0.1), do której należy na tym łączu.
        Liczniki są ustawiane na wartość pseudolosową z zakresu od zera do query interval
        response time. Jeśli licznik był już włączony dla danej grupy multicastowej, to jest on
        tylko wtedy resetowany do wartości losowej, jeśli query interval response time jest
        mniejszy niż pozostała wartość licznika. Jeśli wygaśnie licznik dla grupy, to host musi
        wysłać wiadomość multicast version 2 membership report. Jeśli host odbierze inną
        wiadomość Report od innych hostów, to zatrzymuje licznik i nie wysyła już
        wiadomości Report, w ten sposób unika się powielania raportów dla grupy.
        \item Dodano mechanizm wyboru routera odpytującego (querier) w segmencie sieci
        wykorzystującej wielodostęp. Zostaje nim router, którego adres IP jest najmniejszą
        liczbą. Domyślnie router zakłada, że jest routerem
        odpytującym, ale jak dostanie query od routera z „niższym” adresem IP (na tym
        samym łączu), to przestanie być routerem odpytującym (staje się non-querier). Jeśli
        router non-querier nie słyszy komunikatów query od routera odpytującego przez
        pewien czas (other querier present interval), to staje się routerem odpytującym (ale
        znowu tylko do chwili, gdy dostanie query od routera z niższym adresem IP).
    \end{itemize}

    \subsubsection{IGMPv3}

    Dodano możliwość zgłaszania się do grup z wyspecyfikowaniem adresu
    jednostkowego IPv4 pewnego nadawcy. Można więc zgłosić się do grupy
    na przykład 235.32.32.35, ale z wskazaniem, że interesują nas tylko pakiety od konkretnego
    źródła.


    \subsection{Transmisje grupowe a technologie sieci lokalnych}

    Ethernet daje możliwość adresowania MAC typu multicast. Wykorzystywane są adresy z
    zakresu 01:00:5e:00:00:00 do 01:00:5e:7f:ff:ff. 23 bity adresu IPv4 są wprost wykorzystane w
    adresie MAC.
    Dla IPv4:

    \begin{figure}[H]
        \includegraphics[width=\linewidth]{ipv4.png}
    \end{figure}

    Zatem każdy adres Ethernet multicast jest związany z 32 adresami IPv4 z klasy D (różniącymi
    się na 5 bitach). Trzeba to uwzględnić przy projektowaniu multiemisji. Może się zdarzyć, że
    pewien host będzie otrzymywał ramki zawierające pakiety IPv4 z grupy, której nie jest
    odbiorcą. Jednak pakiet taki zostanie odrzucony po odczytaniu adresu IPv4.


    Przykłady
    $239.20.20.20$ odpowiada adresowi MAC: $01-00-5e-14-14-14$.\\
    $239.10.10.10$ odpowiada adresowi MAC: $01-00-5e-0a-0a-0a$.\\
    Przykład adresów IP multiemisji, które odwzorowane są w ten sam adres MAC:
    Adresy, które różnią się w zapisie bitowym na pozycjach oznaczonych \_ są odwzorowane w
    ten sam adres MAC\\
    1110\_ \_ \_ \_ . \_ xxxxxxx. xxxxxxxx.xxxxxxxx\\
    $224.7.7.7$\\
    $225.7.7.7$\\
    $226.7.7.7$\\
    $227.7.7.7$\\
    …\\
    $239.7.7.7$\\
    $224.135.7.7$\\
    $225.135.7.7$\\
    $226.135.7.7$\\
    $227.135.7.7$\\
    …\\
    $239.135.7.7$\\

    Standardowo przełącznik Ethernet traktuje ramkę z adresem MAC multiemisji jak ramkę z
    adresem jednostkowym MAC, którego nie zna, czyli po otrzymaniu takiej ramki na
    pewnym porcie, przesyła ją do wszystkich pozostałych portów (realizuje broadcast w
    warstwie drugiej). W celu wyeliminowania tego rozgłoszenia stosowany jest mechanizm
    nazywany IGMP snooping lub specjalny protokół o nazwie CGMP.


    \textbf{IGMP Snooping}\\
    IGMP snooping polega na tym, że przełącznik warstwy drugiej „słucha” konwersacji między
    hostami a routerami i analizuje pakiety z komunikatami IGMP (raporty członkostwa w grupie
    membership reports oraz zgłoszenia opuszczenia grupy – membership leaves). Na podstawie
    śledzonych komunikatów IGMP przełącznik aktualizuje swoją tablicę przypisania adresów
    MAC do portów (CAM – Content Addressable Memory) i uwzględnia adresy Ethernet
    multicast.
    To rozwiązanie wymaga jednak odpowiednio wydajnych przełączników, najlepiej z
    dołączonym specjalnym sprzętowym modułem (ASIC) do analizy komunikatów IGMP.


    \textbf{Protokół CGMP}\\
    Cisco Group Management Protocol (CGMP) jest oparty o model klientserver, gdzie router jest serwerem CGMP a przełącznik jest klientem. Na routerze i
    przełączniku działa oprogramowanie realizujące CGMP. Router tłumaczy komunikaty IGMP
    (membership report i leave wędrujący od hosta do routera) na komendy CGMP, które są
    przesyłane do przełącznika i tam wykorzystane do modyfikowania tablicy adresów MAC.
    Routery wykorzystują ogólnie znany adres multicast MAC (01:00:0c:dd:dd:dd) do przesyłania
    komend do przełączników.


    \textbf{Protokół PIM}\\
    Protokół routowania multiemisji (Protocol Independent Multicast).
    W odróżnieniu od protokołów routowania jednostkowego opartych na grafach,
    protokoły dla multicastu oparte są o drzewa dystrybucji. Drzewa te określają ścieżki od
    źródła do odbiorników (odbiorców).
    Wykorzystywane są dwa rodzaje drzew:
    \begin{itemize}
        \item Source trees (drzewa źródłowe)
        \begin{itemize}
            \item oddzielne drzewo jest budowane
            od każdego źródła do odbiorników
            \item Source tree określa najkrótszą ścieżkę od źródła do
            odbiorcy, drzewo takie jest też nazywane Shortest Path Tree (SPT)
            \item  Routery zapamiętują
            informacje o parach (Source IP address, Group IP address). W przypadku, gdy na pewnym
            obszarze (z wieloma odbiorcami) takich par jest dużo, czyli gdy grupy zawierają wiele źródeł i
            jest tych grup dużo, obsługa multiemisji może zajmować sporo zasobów pamięciowych
            routera.
        \end{itemize}
        \item Shared trees (drzewa współdzielone).
        \begin{itemize}
            \item ścieżki od różnych źródeł zawierają
            wspólną część – drzewo współdzielone, którego korzeń jest routerem służącym jako punkt
            spotkań – Randezvous Point (RP)
            \item  Źródła wysyłają pakiety do RP, skąd przesyłane są dalej do
            odbiorców przez drzewo współdzielone.
            \item Ze względu na wykorzystanie wspólnego punktu
            spotkań, ścieżki od źródła do odbiorcy mogą być nieoptymalne.
            \item Zaletą jest jednak znacznie
            mniejsze wykorzystanie zasobów routerów.
        \end{itemize}
    \end{itemize}


    Typy PIM:
    \begin{itemize}
        \item dense mode (DM)\\
        Routery kierują ruch grupowy (multicast) w sposób zalewowy
        (flooding) do wszystkich sieci na pewnym obszarze, a następnie obcinają te przepływy
        (krawędzie grafu), dla których nie ma odbiorników wykorzystując okresowy działający
        mechanizm nazywany flood-and-prune. Generalnie założenie jest tutaj takie, że odbiorcy są
        prawie wszędzie w rozpatrywanym obszarze.
        \begin{itemize}
            \item Faza Flood\\
            multicast jest kierowany do wszystkich interfejsów (w wielu
            routerach) z wyjątkiem tych, które prowadzą najkrótszą ścieżką do źródła. Nazwijmy te
            interfejsy non-RPF. Wyznaczenie interfejsów prowadzących najkrótszą ścieżką do źródła
            (nazwijmy je RPF) następuje dzięki pracy zwykłego protokołu routowania unicastowego (np.
            EIGRP, OSPF)
            \item Faza Prune\\
            routery, które nie mają odbiorców wysyłają do interfejsów RPF i
            non-RPF komunikaty Prune, które powodują zablokowanie przesyłania ruchu grupowego do
            fragmentów sieci, w których przesyłanie grupowe nie powinno być realizowane. Router,
            który ma odbiorców ruchu grupowego wysyła komunikaty Prune do tych interfejsów nonRPF, z których nie powinien nadchodzić ruch grupowy (bo nie są na najkrótszej ścieżce do
            źródła).
            \item Po kilku minutach (w PIM DM standardowo 3 min.) fazy Flood oraz Prune są realizowane
            ponownie. W przypadku, jeśli w pewnym miejscu, do którego nie dociera ruch grupowy (bo
            zadziałała faza Prune) pojawia się odbiorca, może być przesłany odpowiedni komunikat
            (Graft), który szybko przywraca ruch grupowy w odpowiednim fragmencie sieci (nie trzeba
            czekać 3 min.).
        \end{itemize}

        \item sparse mode (SM)\\
        drzewa dystrybucji są budowane z wykorzystaniem jawnych
        mechanizmów połączenia (explicit join tree) kierowanych od routerów, do których
        podłączeni są bezpośrednio odbiorcy do (w kierunku) źródła.

    \end{itemize}


    Protokoły routowania multicast wykorzystują mechanizm Reverse Path Forwarding do
    utworzenia najkrótszych ścieżek od źródła do odbiorcy i do zapobiegania pętlom. Można
    powiedzieć, że jednostkowe protokoły routowania skupiają się na tym, gdzie jest
    odbiorca, natomiast protokoły multicast skupiają
    się na tym, gdzie jest źródło .


    \begin{itemize}
        \item wykorzystuje shared distribution trees, chociaż może również wykorzystywać
        drzewa SPT (przełącza się na SPT).
        \item zwykle wykorzystuje RP (Randezvous Point). Źródła rejestrują się w RP (muszą go
        znać) i wysyłają ruch multicastowy do RP przy pomocy pakietów unicastowych.
        \item Odbiorcy
        przyłączają się do grupy wykorzystując swój DR (designated router) na swoim łączu. Router
        ten musi znać adres RP i wysyła komunikat dołączenia grupy w kierunku RP,
        wykorzystując informację z normalnego routowania unicastowego.
        \item Komunikat wędruje metodą hop-by-hop aż dotrze do RP, budując tym samym
        gałąź drzewa shared tree. W przypadku, jeśli węzeł wykryje, że zna lepszą ścieżkę do źródła
        niż przez shared tree, przełącza się na SPT.
    \end{itemize}

    Są różne usprawnienia/modyfikacje, np. Bidirectional PIM dla ruchu typu many-to-many.
    Istnieje możliwość skonfigurowania mechanizmu automatycznego wyboru RP.
    Jest też wersja PIM Sparse-Dense-Mode. W tej wersji PIM, jeśli nie zostanie wykryty RP
    (można skonfigurować automatyczny wybór RP) ani nie zostanie skonfigurowany ręcznie,
    PIM przechodzi do trybu PIM DM. Zalecaną wersją PIM jest właśnie PIM Sparse-DenseMode.


\end{document}