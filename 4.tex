\documentclass[../main.tex]{subfiles}


\begin{document}

    \begin{table}[H]
        \begin{center}
            \begin{tabular}{ | p{.2\textwidth} | p{.1\textwidth} | p{.1\textwidth} | p{.15\textwidth} | p{.2\textwidth} | p{.2\textwidth} | }
                \hline
                Preambuła & src MAC & dst MAC & Typ danych & 46-1500 Dane & CRC\\
                \hline
                żeby się karty

                sieciowe

                zsynchronizowały & & & & tu jest pakiet IP i

                segment TCP & suma kontrolna\\
                \hline
                warstwa 1 & \multicolumn{5}{c}{warstwa 2 }|\\
                \hline
            \end{tabular}
        \end{center}
    \end{table}

    \subsection{Dostęp do nośnika}

    \textbf{Dla Ethernet I}: Dostęp do nośnika realizowany był na zasadzie \textbf{CSMA} (Carrier Sense, Multiple Access – wielodostęp do nośnika z badaniem stanu nośnika).

    \textbf{Dla Ethernet II}: Dostęp do nośnika realizowany był na zasadzie \textbf{CSMA/CD} (Carrier Sense, Multiple Access with Collision Detection – wielodostęp do nośnika z badaniem stanu oraz wykrywaniem kolizji).
    Teraz już nie ma problemu z kolizjami, bo mamy switche.

    Kabel cross, jak łączymy bez switcha dwa urządzenia.
    Kabel prosty, jak jest switch
    Teraz sobie urządzenia wykrywają czy są dobrze połączone i same korygują żeby było ok.



\end{document}