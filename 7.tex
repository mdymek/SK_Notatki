\documentclass[../main.tex]{subfiles}
\begin{document}

    Multicast – transmisja grupowa, multiemisja.
    \begin{itemize}
        \item Wysłanie jednego pakietu ze źródła do wielu miejsc docelowych. Pakiety są kopiowane
        w routerach i przełącznikach warstwy drugiej.
        \item Mniejsze obciążenie
        sieci, większa skalowalność w stosunku do unicastu
        \item Schematy jeden-do-wielu, wiele-do-wielu.
        \item Komunikaty w większości protokołów routowania mają zarezerwowane adresy multiemisji.
        \begin{itemize}
            \item 224.0.0.1 – wszystkie komputery uczestniczące w transmisji grupowej (również routery) w
            segmencie sieci lokalnej.
            \item 224.0.0.2 – wszystkie routery uczestniczące w transmisji grupowej (multicast routers) w
            segmencie sieci lokalnej.
            \item 224.0.0.4 – routery DVMRP.
            \item 224.0.0.5 – wszystkie routery OSPF.
            \item 224.0.0.6 – routery DR OSPF.
            \item 224.0.0.9 – routery RIPv2 (RIPv1 wykorzystuje rozgłoszenie – broadcast, nie multicast).
            \item 224.0.0.0 – 239.255.255.255 - klasa D adresów IPv4
        \end{itemize}
        \item Aby uczestniczyć w transmisji grupowej, komputer musi sprawdzać określone adresy w
        przychodzących pakietach (IP) i generalnie w ramkach (MAC).
        \item Transmisja grupowa odbywa się z wykorzystaniem różnych mechanizmów i protokołów.

\end{itemize}

    \subsection{IGMP - Internet Group Management Protocol}
    \begin{itemize}
        \item wykorzystywany do dynamicznego rejestrowania/wyrejestrowania
        odbiornika w routerze
        \item komunikaty IGMP są przesyłane w pakietach IP z adresem docelowym typu multicast i
        ustawioną wartością TTL na 1.
    \end{itemize}


    \subsubsection{IGMPv1}

    Są dwa typy komunikatów:
    \begin{itemize}
        \item Membership query (general membership query), wysyłany jest okresowo (co
        kilkadziesiąt sekund) przez routery na wszystkiue komputery.
        \item Membership report służy do zgłoszenia się jako odbiorca pakietów wysyłanych na ten adres; membership
        report wysyłany jest też w odpowiedzi na membership query.
    \end{itemize}


    Host po otrzymaniu membership query czeka pewien pseudolosowy czas z
    zakresu od 0 do 10 sekund) i wysyła membership report. Jeśli w tym pseudolosowym czasie
    host usłyszy membership report od innego hosta, to nie wysyła swojego raportu.

    W IGMPv1 host „po cichu” opuszcza grupę. Jeśli router nie dostanie raportu w odpowiedzi
    na trzy kolejne membership query,
    router usuwa grupę z tablicy multicastu i przestaje przesyłać pakiety kierowane do tej grupy.

    \subsubsection{IGMPv2}

    W IGMPv2 są cztery typy komunikatów:
    \begin{itemize}
        \item Membership query
        \item Version 1 membership report
        \item Version 2 membership report
        \item Leave group
    \end{itemize}


    Ważne zmiany w porównaniu do wersji pierwszej:
    \begin{itemize}
        \item Membership query może być typu group-specific query.
        \item Leave group message – komunikat o opuszczeniu grupy, wysyłany jest na adres
        224.0.0.2 (wszystkie routery multicast na łączu).
        \item Dodano do zapytań IGMP określenie czasu query-interval response time , jaki mają uczestnicy na wysłanie raportu, czas ten jest określany przez
        wysyłającego zapytanie.
        \item Dodano mechanizm wyboru routera odpytującego (querier) w segmencie sieci
        wykorzystującej wielodostęp. Zostaje nim router, którego adres IP jest najmniejszą
        liczbą. Domyślnie router zakłada, że jest routerem
        odpytującym, ale jak dostanie query od routera z „niższym” adresem IP, to przestanie być routerem odpytującym. Jeśli
        router non-querier nie słyszy komunikatów query od routera odpytującego przez
        pewien czas, to staje się routerem odpytującym.
    \end{itemize}

    \subsubsection{IGMPv3}

    Dodano możliwość zgłaszania się do grup z wyspecyfikowaniem adresu
    jednostkowego IPv4 pewnego nadawcy.


    \subsection{Transmisje grupowe a technologie sieci lokalnych}

    Ethernet daje możliwość adresowania MAC typu multicast. Wykorzystywane są adresy z
    zakresu 01:00:5e:00:00:00 do 01:00:5e:7f:ff:ff. 23 bity adresu IPv4 są wprost wykorzystane w
    adresie MAC.

    Zatem każdy adres Ethernet multicast jest związany z 32 adresami IPv4 z klasy D (różniącymi
    się na 5 bitach).

    Przykłady
    $239.20.20.20$ odpowiada adresowi MAC: $01-00-5e-14-14-14$.\\
    $239.10.10.10$ odpowiada adresowi MAC: $01-00-5e-0a-0a-0a$.\\


    \textbf{IGMP Snooping}\\
    IGMP snooping polega na tym, że przełącznik warstwy drugiej „słucha” konwersacji między
    hostami a routerami i analizuje pakiety z komunikatami IGMP (raporty członkostwa w grupie
    membership reports oraz zgłoszenia opuszczenia grupy – membership leaves). Na podstawie
    śledzonych komunikatów IGMP przełącznik aktualizuje swoją tablicę przypisania adresów
    MAC do portów (CAM – Content Addressable Memory) i uwzględnia adresy Ethernet
    multicast.
    To rozwiązanie wymaga jednak odpowiednio wydajnych przełączników, najlepiej z
    dołączonym specjalnym sprzętowym modułem (ASIC) do analizy komunikatów IGMP.


    \textbf{Protokół CGMP}\\
    Switch “słucha” konwersacji między hostami a routerami i analizuje pakiety z IGMP, na tej podstawie aktualizuje tablicę MAC portów i wysyła do komputerów to co chcą słuchać


\end{document}