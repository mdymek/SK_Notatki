\documentclass[../main.tex]{subfiles}
\begin{document}

    \textbf{Multicast – transmisja grupowa, multiemisja.}
    \begin{itemize}
        \item Wysłanie jednego pakietu ze źródła do wielu miejsc docelowych. Pakiety są kopiowane
        w routerach i przełącznikach warstwy drugiej.
        \item Mniejsze obciążenie
        sieci, większa skalowalność w stosunku do unicastu.
        \item Schematy jeden-do-wielu, wiele-do-wielu.
        \item Komunikaty w większości protokołów routowania mają zarezerwowane adresy multiemisji.
        \item Aby uczestniczyć w transmisji grupowej, komputer musi sprawdzać określone adresy w
        przychodzących pakietach (IP) i generalnie w ramkach (MAC).
        \item Transmisja grupowa odbywa się z wykorzystaniem różnych mechanizmów i protokołów.
\end{itemize}

    \subsection{IGMP - Internet Group Management Protocol}
    \begin{itemize}
        \item wykorzystywany do dynamicznego rejestrowania/wyrejestrowania
        odbiornika w routerze
        \item komunikaty IGMP są przesyłane w pakietach IP z adresem docelowym typu multicast i
        ustawioną wartością TTL na 1.
    \end{itemize}


    \subsubsection{IGMPv1}
    Są \textbf{dwa} typy komunikatów:
    \begin{itemize}
        \item \textbf{Membership query}, wysyłany okresowo (co
        kilkadziesiąt sekund) przez routery na wszystkiue komputery.
        \item \textbf{Membership report} służy do zgłoszenia się jako odbiorca pakietów wysyłanych na ten adres; wysyłany jest też w odpowiedzi na membership query.
    \end{itemize}


    Host po otrzymaniu membership query czeka pewien pseudolosowy czas i wysyła membership report. Jeśli w tym pseudolosowym czasie
    host usłyszy membership report od innego hosta, to nie wysyła swojego raportu.

    W IGMPv1 \textbf{host „po cichu” opuszcza grupę}. Jeśli router nie dostanie raportu w odpowiedzi
    na \textbf{trzy kolejne} membership query,
    router usuwa grupę z tablicy multicastu i przestaje przesyłać pakiety kierowane do tej grupy.

    \subsubsection{IGMPv2}

    W IGMPv2 są \textbf{cztery} typy komunikatów:
    \begin{itemize}
        \item \textbf{Membership query}
        \item \textbf{Version 1 membership report}
        \item \textbf{Version 2 membership report}
        \item \textbf{Leave group}
    \end{itemize}


    Ważne zmiany w porównaniu do wersji pierwszej:
    \begin{itemize}
        \item Membership query może być typu \textbf{group-specific query}.
        \item Leave group message – \textbf{komunikat o opuszczeniu grupy}, wysyłany jest na adres
        224.0.0.2 (wszystkie routery multicast na łączu).
        \item Dodano do zapytań IGMP określenie czasu \textbf{query-interval response time}, jaki mają uczestnicy na wysłanie raportu, czas ten jest określany przez
        wysyłającego zapytanie.
        \item Dodano mechanizm wyboru routera odpytującego (\textbf{querier}) w segmencie sieci
        wykorzystującej wielodostęp. Zostaje nim router, którego adres IP jest najmniejszą
        liczbą.
    \end{itemize}

    \subsubsection{IGMPv3}

    Dodano możliwość zgłaszania się do grup z \textbf{wyspecyfikowaniem adresu
    jednostkowego} IPv4 pewnego nadawcy.


    \subsection{Transmisje grupowe a technologie sieci lokalnych}

    Ethernet daje możliwość adresowania \textbf{MAC typu multicast}. Wykorzystywane są adresy z
    zakresu 01:00:5e:00:00:00 do 01:00:5e:7f:ff:ff. \textbf{23 bity adresu IPv4 są wprost wykorzystane w
    adresie MAC}.

    Zatem każdy adres Ethernet multicast jest związany z 32 adresami IPv4 z klasy D (różniącymi
    się na 5 bitach).

    Przykłady
    $239.20.20.20$ odpowiada adresowi MAC: $01-00-5e-14-14-14$.\\
    $239.10.10.10$ odpowiada adresowi MAC: $01-00-5e-0a-0a-0a$.\\


    \textbf{IGMP Snooping}\\
    IGMP snooping polega na tym, że \textbf{przełącznik warstwy drugiej „słucha”} konwersacji między
    hostami a routerami i analizuje pakiety z komunikatami IGMP (raporty członkostwa w grupie
    membership reports oraz zgłoszenia opuszczenia grupy – membership leaves). Na podstawie
    śledzonych komunikatów IGMP przełącznik \textbf{aktualizuje swoją tablicę} przypisania adresów
    MAC do portów i uwzględnia adresy Ethernet
    multicast.
    To rozwiązanie wymaga jednak odpowiednio wydajnych przełączników.

    \textbf{Protokół CGMP}\\
    Switch “słucha” konwersacji między hostami a routerami i analizuje pakiety z IGMP, na tej podstawie aktualizuje tablicę MAC portów i wysyła do komputerów to co chcą słuchać


\end{document}