\documentclass[../main.tex]{subfiles}
\begin{document}

    \subsection{System autonomiczny (AS)}
    \begin{itemize}
        \item Grupa złożona z jednego lub większej liczby prefiksów
        sieci należących do jednego lub więcej operatorów, która to grupa ma jedną jasno
        zdefiniowaną politykę routowania.
        \item Numery AS 2-bajtowe, lub 4-bajtowe.
        \item W EIGRP należy określić identyfikator procesu - AS number. Nie musi być unikalny, nie jest
        przekazywany do BG4.
    \end{itemize}


    \subsection{Protokół IGRP}
    \begin{itemize}
        \item protokół wektora odległości
        \item wymaga podania numeru AS przy konfiguracji - numeru procesu IGRP. Musi być taki sam
        we wszystkich routerach na danym obszarze z komunikacją IGRP
        \item metryka 24-bitowa w IGRP jest tworzona na podstawie wartości metryk cząstkowych oraz zmiennych
        określających wagę każdej użytej metryki.
        \begin{itemize}
            \item Szerokość pasma (bandwidth); oznacza liczbę bitów, jakie może transmitować w
            jednostce czasu dana technologia (patrz też objaśnienia w oddzielnym pliku).
            \item Opóźnienie (delay) – czas wędrówki pakietu od źródła do celu; wartości od 1 do 224, przy
            czym 1 oznacza 10 mikrosekund.
            \item Obciążenie (load); wartość od 1 do 255. 1 oznacza sieć najmniej obciążoną, 255
            najbardziej obciążoną.
            \item Niezawodność (reliability); wartość od 1 do 255. Wartość liczona jest jako swoisty
            „procent” pakietów, które dotarły do następnego routera, przy czym liczba 255 oznacza
            100\%.
        \end{itemize}
        $metric = (K_1 * bandwidth + \frac{(K_2*bandwidth)}{(256 - load)} + K_3 * delay) *
        \frac{K_5}{(reliability + K_4)}$
        Standardowo $K_1 == K_3 == 1, K_2 == K_4 == K_5 == 0$.
        \item przesyłane są również wartości MTU (Maximum Transfer Unit) – najmniejsze MTU na trasie do sieci oraz liczba skoków
        \item można definiować największą dopuszczalną liczbę skoków (standardowo=100) powyżej której trasa jest uznawana za nieosiągalną
        \item przechowywanych jest kilka optymalnych tras do pewnego miejsca docelowego, mogą być przechowywane informacje o trasach nieoptymalnych
        \item przeprowadzane równoważenie obciążenia (load balancing). Przez wszystkie
        trasy wiodące do pewnego celu są przesyłane datagramy, przy czym liczba datagramów
        przesyłanych przez pewną trasę jest odwrotnie proporcjonalna do metryki końcowej tej
        trasy)
        \item trzy typy tras:
        \begin{itemize}
            \item \textbf{wewnętrzne} - do podsieci dołączonych bezpośrednio do routera
            \item \textbf{systemowe} - do sieci w ramach tego samego systemu autonomicznego
            \item \textbf{zewnętrzne} - do innych systemów autonomicznych
        \end{itemize}
        \textbf{NIE ma trasy domyślnej 0.0.0.0}. Są trasy zewnętrzne, przez tzw. 'router of last resort'
        wybierane jeśli nie znaleziono żadnej innej.
        \item możliwia równoważenie obciążeń – przesyłanie pakietów do tego samego miejsca
        docelowego rożnymi trasami o tym samym lub nawet większym koszcie. Wymaga to
        zdefiniowania parametru nazywanego wariancją (variance), który określa ile razy gorsze
        trasy (pod względem metryki) mają być użyte.
        \item nie obsługuje routowania bezklasowego ani VLMS.
    \end{itemize}


    \textbf{Zastosowane mechanizmy zapobiegania niekorzystnym zjawiskom}
    \begin{itemize}
        \item holddown-timers\\
        Wektory odległości są rozgłaszane co 90 sekund (update timer). Trasa jest uważana za
        niepoprawną, jeśli nie nadeszły z niej trzy kolejne rozgłoszenia (invalid timer - 270 sekund).
        Hold-down timer – 280s. Flush timer – 630 s.
        \item dzielony horyzont (split horizon),
        \item zatruwanie tras (poison reverse),
        \item natychmiastowe aktualizacje (triggered updates).
    \end{itemize}



    \subsection{Protokół EIGRP}
    \begin{itemize}
        \item protokół wektora odległości
        \item nazywany protokołem hybrydowym, łączący zalety protokołów typu wektora odległości i
        typu stanu łącza
        \item obsługuje adresowanie bezklasowe Classless Inter-Domain Routing oraz maski
        podsieci Variable Length Subnet Mask
        \item konfiguracja EIGRP wymaga określenia numeru AS - numeru procesu EIGRP; taki sam w komunikujących
        się routerach
        \item IGRP i EIGRP mogą ze sobą współpracować, jeśli mają ten sam numer; nastąpi przeliczenie metryki;
        trasa z IGRP jest traktowana jak trasa zewnętrzna
        \item metryka 32 bitowa; aktualizacje zawierają liczbę skoków dla
        trasy, jednak liczba skoków nie jest brana pod uwagę przy wyliczaniu metryki
    \end{itemize}


    \textbf{Kluczowe technologie i idee wykorzystane w EIGRP}
    \begin{itemize}
        \item Wykrywanie sąsiadów (neighbors Discovery).
        \item Reliable Transport Protocol (RTP) – niezawodny protokół warstwy transportu.
        \item Diffusing Update Algorithm DUAL, maszyna skończenie stanowa DUAL (DUAL finie-state
        machine).
        \item Wysyłanie aktualizacji tylko po wykryciu nowego sąsiada i w przypadku wystąpienia
        zmiany.
        \item Sprawdzanie łącza do sąsiada na podstawie krótkich komunikatów HELLO wysyłanych
        okresowo (standardowo co 5 sekund, dla łączy szeregowych co 60 sekund).
        \item Budowa modularna, praca z różnymi protokołami routowalnymi (AppleTalk, IPX,
        możliwość obsługi nowych protokołów).
    \end{itemize}

    \textbf{Wybrane zalety EIGRP}
    \begin{itemize}
        \item Minimalne zużycie szerokości pasma gdy sieć jest stabilna. W czasie normalnego
        stabilnego działania sieci jedynymi wymienianymi pakietami pomiędzy węzłami EIGRP są
        pakiety HELLO (handshake).
        \item Wydajne wykorzystanie szerokości pasma w czasie uzyskiwania zbieżności. Po zmianie
        propagowane są jedynie zmiany, nie całe wektory odległości. Po wykryciu sąsiada
        uaktualnienie wysyłane jest tylko do niego (unicast).
        \item Szybka zbieżność po wykryciu zmiany w sieci. Routery EIGRP zapamiętują wszystkie trasy
        przesłane przez sąsiadów. Ponadto wśród tych tras są od razu wyznaczane trasy
        zastępcze, nie zawierające pętli, o ile takie są (według podanych niżej reguł).
        \item Niezależność od protokołów routowalnych.
        \item Obsługa CIDR, VLSM.

    \end{itemize}


    \textbf{Charakterystyka tablic EIGRP}
    \begin{itemize}
        \item \textbf{Tablica sąsiadów} - zawiera dane o sąsiadach, z którymi są wymieniane
        informacje o sieciach; zawiera adresy IP i interfejsy, kiedy nastąpił jakiś kontakt z
        sąsiadem. Czas hold time określa jak długo można uznawać trasę wiodącą przez pewnego
        sąsiada za poprawną, jeśli router nie dostał od tego sąsiada kilku kolejnych pakietów
        HELLO. Standardowo hold time jest równy 3 * okres wysyłania pakietów HELLO.
        \item \textbf{Tablica topologii} - zawiera wszystkie trasy zgłoszone przez
        sąsiadów. Zawiera metrykę całkowitą trasy, reported distance i feasible distance.
        \item \textbf{Tablica routowania} - przechowuje trasy o najniższym koszcie (do 6 tras
        alternatywnych); można stosować mechanizm równoważenia obciążeń z wykorzystaniem wariancji.
    \end{itemize}

    EIGRP wykorzystuje specjalny niezawodny protokół w warstwie transportu – \textbf{Reliable
    Transport Protocol}. RTP umożliwia wykorzystanie transmisji grupowej (multicast) lub
    jednostkowej (unicast).
    \begin{itemize}
        \item Aktualizacje są przesyłane niezawodnie (z wykorzystaniem numeru sekwencji i mechanizmu
        potwierdzania) na adres grupowy 224.0.0.10. Potwierdzenia są przesyłane na adres
        jednostkowy (unicast). Jeśli potwierdzenie z określonym numerem sekwencji nie nadejdzie w
        czasie RTO (Retransmission TimeOut), pakiet z aktualizacją jest retransmitowany, tym razem
        na adres jednostkowy.
        \item Zwykłe pakiety HELLO oraz potwierdzenia nie są potwierdzane.
        \item DUAL jest używany do wyznaczenia sukcesorów i tzw. wykonalnych
        sukcesorów określających trasy zapasowe. W przypadku utraty pewnej trasy (uszkodzenia) router może natychmiast wyznaczyć
        niezapętloną trasę zastępczą (jeśli jest wyznaczony FS).  Gdyby się zdarzyło, że nie ma
        informacji o trasie zastępczej, to router prosi sąsiadów o odnalezienie takiej trasy, jeśli
        sąsiedzi nie znajdują, to odpytują dalej. Zapytanie rozchodzi się (dyfunduje) coraz dalej, stąd
        nazwa DUAL (Diffusion Algorithm).
        \item Mechanizm wyznaczania tras zapasowych zapewnia, że nie ma w nich pętli routowania.
    \end{itemize}


\end{document}