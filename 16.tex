\documentclass[../main.tex]{subfiles}


\begin{document}

    \subsection{Szyfrowanie z kluczem}
    \begin{itemize}
        \item liczba lub kilka liczb, składająca się z kilkudziesięciu do kilku tysięcy bitów
        \item służy do szyfrowania i odszyfrowywania rzeczy
        \item różne algorytmy szyfrowania
        \begin{itemize}
            \item \textbf{Szyfrowanie z kluczem symetrycznym}\\
            \begin{itemize}
                \item \textbf{Teoretycznie możliwy do złamania} brute forcem (w praktyce niezbyt).
                \item Szyfrowanie dużych porcji danych przy użyciu jednego klucza ułatwia złamanie
                szyfru, dlatego klucz symetryczny \textbf{powinien być zmieniany}.
                \item Może być przesłany zaszyfrowany przy pomocy techniki z kluczem
                publicznym i prywatnym.
                \item Można generować oddzielne klucze sesji i szyfrować je
                wcześniej uzgodnionym tajnym kluczem symetrycznym. Klucze symetryczne mogą być też
                zmieniane co określony czas.
                \item Algorytmy szyfrujące używające klucza symetrycznego: DES, 3DES, RC (WiFi/WPA), AES (WPA2)
            \end{itemize}
            \item \textbf{Szyfrowanie z kluczem asymetrycznym}
            \begin{itemize}
                \item Szyfrowanie i odszyfrowanie jest tu realizowane przy pomocy pary kluczy - \textbf{prywatnego
                i publicznego}. Jeden szyfruje, drugi odszyfrowywuje.
                \item \textbf{Odgadnięcie} jednego z kluczy \textbf{praktycznie niemożliwe} nawet przy znajomości drugiego.
                \item Szyfrujemy coś czyimś kluczem publicznym, by tylko ten ktoś mógł to odszyfrować (swoim kluczem prywatnym).
                \item Wielokrotnie \textbf{kosztowniejsze czasowo} od szyfrowania z kluczem symetrycznym.
                \item Używane do uzgodnienia kluczy symetrycznych.
                \item Algorytm RSA.
            \end{itemize}
        \end{itemize}

    \end{itemize}

    \subsubsection{Skrót (hash)}
    \begin{itemize}
        \item \textbf{Skrót wiadomości} w podpisach cyfrowych, tworzony za pomocą funkcji haszujacej.
        \item 128 bitów (MD5), 160 bitów(SHA-1), 224-512 bitów (rodzina SHA-2)
        \item Jeśli w oryginalnej wiadomości (pliku)
        zmieniony \textbf{zostanie chociaż jeden bit, to skrót będzie zupełnie inny} niż ten, który został
        utworzony przed zmianą.
        \item \textbf{Algorytmy haszujące są deterministyczne}.
        \item \textbf{Odtworzenie} oryginalnej wiadomości ze skrótu jest \textbf{prawie niemożliwe}.
    \end{itemize}


    \subsubsection{Podpis cyfrowy}
    \begin{itemize}
        \item Zaszyfrowanie kluczem prywatnym daje \textbf{gwarancję}, że zaszyfrowana wiadomość \textbf{pochodzi z
        odpowiedniego źródła}.
        \item Samej podpisywanej wiadomości nie musi się szyfrować. Generowany jest jej skrót
        i ten \textbf{skrót jest szyfrowany} z wykorzystaniem klucza
        \textbf{prywatnego} osoby \textbf{podpisującej}. Zaszyfrowany skrót stanowi podpis cyfrowy.
        Niezaszyfrowana wiadomość może być przesłana jawnie razem z zaszyfrowanym skrótem
        (czyli podpisem cyfrowym).
        \item Odbiorca \textbf{odszyfrowuje} skrót używając klucza
        \textbf{publicznego nadawcy}. Potem tworzy skrót wiadomości używając tej samej funkcji haszującej. Jeśli wyniki
        obu operacji są identyczne, to znaczy, że wiadomość na pewno
        podpisał określony nadawca, a ponadto nikt po tej wiadomości nie zmienił już po podpisaniu.
    \end{itemize}

    \subsubsection{Klucze publiczne i prywatne, infrastruktura kluczy publicznych}
    \begin{itemize}
        \item Klucze mogą być generowane na komputerze
        lokalnym przy pomocy odpowiedniego oprogramowania i powinny być podpisane przez
        jakieś centrum certyfikacyjne.
        \item Centrum certyfikacyjne (CA) wydaje tzw. certyfikaty cyfrowe zawierające m.in:
        \begin{itemize}
            \item Identyfikator osoby/firmy/obiektu
            \item Identyfikator CA, który wydał certyfikat
            \item Numer identyfikacyjny certyfikatu
            \item Cel stosowania (np. podpisywanie bezpiecznych stron WWW albo podpisywanie listów elektronicznych)
            \item Wartość klucza publicznego
            \item Okres ważności
            \item Podpis cyfrowy wydawcy
        \end{itemize}
        \item Jeśli ufamy danemu CA, to ufamy, że zawarty w certyfikacie klucz publiczny jest rzeczywiście prawdziwy.
        W systemach operacyjnych oraz różnych programach jest wpisana lista zaufanych CA.
        Zarządzanie centrum certyfikacyjnym jest realizowane na przez konsolę MMC.
        \item Niezależnym standardem opisującym tworzenie kluczy, rejestrowanie i wykorzystywanie
        certyfikatów jest PGP (Pretty Good Privacy). Powstał standard Open PGP.

    \end{itemize}

\end{document}