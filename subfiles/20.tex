\documentclass[../main.tex]{subfiles}

\begin{document}

    \textbf{Gniazda} - to \textbf{abstrakcyjne} mechanizmy umożliwiające wykonywanie systemowych
    \textbf{funkcji wejścia–wyjścia} w odniesieniu do sieci. Umożliwiają między innymi
    \textbf{przesyłanie danych między procesami} działającymi na komputerach w sieci z wykorzystaniem
    połączeń TCP lub protokołu UDP, przy czym same operacje wysyłania i odbierania danych
    przypominają zwykłe operacje zapisywania i odczytu z pliku.


    \begin{itemize}
        \item \textbf{Iteracyjne} - takie które kolejkują klientów.
        \item \textbf{Współbieżne} - starają się od razu wszystko równolegle obsłużyć.
    \end{itemize}

    Programowanie w skrócie:
    \begin{itemize}
        \item Struktura \textbf{socaddr\_in}
        \begin{itemize}
            \item \textbf{długość} struktury,
            \item \textbf{typ adresu} (AF\_INET),
            \item \textbf{nr portu},
            \item struktura z \textbf{adresem} urządzenia.
        \end{itemize}
        \item Klient uzupełnia socaddr\_in danymi serwera.
        \item Wywołuje funkcję \textbf{socket} podając jej:
        \begin{itemize}
            \item \textbf{typ adresu} (AF\_INET),
            \item \textbf{typ gniazda} - datagramowe (UDP) lub streamowe (TCP)
        \end{itemize}
        \item \textbf{(przy TCP)} Przekazuje to co zwrócił socket (\textbf{deskryptor gniazda}) i strukturę socaddr\_in do funkcji \textbf{connect} i tworzy połączenie
        \item Używa socketa jak \textbf{pliku} - zwykłe pisanie i czytanie.
    \end{itemize}

\end{document}