\documentclass[../main.tex]{subfiles}


\begin{document}

    \textbf{Metryką jest szybkość łącza.}
    \begin{itemize}
        \item informacje rozsyłane \textbf{do wszystkich} węzłów
        \item rozsyłana tylko część tablicy routingu o \textbf{bezpośrednio połączonych} sieciach
        \item każdy router buduje \textbf{pełną mapę topologii}, wylicza najlepsze ścieżki (Dijkstra)
    \end{itemize}

    \textbf{OSPF}
    \begin{itemize}
        \item protokół \textbf{bezklasowy}
        \item umożliwia \textbf{uwierzytelnienie}
        \item \textbf{szybko uzyskuje zbieżność}, trigger updates
        \item \textbf{flooding} do utworzenia map topologii
    \end{itemize}

    \textbf{Nagłówek OSPF}: typ pakietu, id routera, id obszaru.

    \textbf{Pakiety OSPF}
    \begin{itemize}
        \item \textbf{Hello} - nawiązywanie sąsiedztwa.
        \item \textbf{DBD} - skrócona lista bazy danych łącze stan
        \item \textbf{LSR} - żądanie dodatkowych informacji o wpisie z DBD
        \item \textbf{LSU} - aktualizacja będąca odpowiedzą na LSR
        \item \textbf{LSA} - potiwerdzenie odebrania LSU
    \end{itemize}


    \textbf{Rodzaje routerów}
    \begin{itemize}
        \item \textbf{Router desygnowany} - hosty mają z nim relację przyległości, z pozostałymi sąsiadami nie (on reaguje na zmiany, hosty na niego).
        \item Routery wewnętrzne, brzegowe, szkieletowe, brzegowe AS.
        \item \textbf{Gateway of last resort} - router do którego idziemy kiedy nie mamy trasy.
    \end{itemize}


    Wszystkie obszary połączone do Area 0. \textbf{Rodzaje obszarów}:
    \begin{itemize}
        \item “normalne” - bez stuba
        \item \textbf{Stub Area} – do takiego obszaru NIE są wprowadzane trasy zewnętrzne, natomiast sumy tras z innych obszarów są wprowadzane.
        \item \textbf{Totally Stubby Area} – do takiego obszaru nie są wprowadzane ani trasy zewnętrzne, ani sumy tras z innych obszarów OSPF. Wyjście z takiego obszaru jest tylko przez trasę domyślną
        \item \textbf{Not So Stubby Area (NSSA)} – obszar Stub, do którego wprowadzane są pewne (na ogół nieliczne) trasy zewnętrzne, które następnie przekazywane są do innych obszarów tak jak sumy tras.
        \item \textbf{Not So Stubby Totally Stubby Area} – obszar połączenie NSSA i Totally Stabby Area. Routery ABR na granicach różnych obszarów powinny być odpowiednio skonfigurowane, co stanowi dodatkową trudność w konfigurowaniu OSPF.
    \end{itemize}
    Trasy zewnętrzne - z innych protokołów routingu.


\end{document}