\documentclass[../main.tex]{subfiles}


\begin{document}
    \textbf{ARP (Address Resolution Protocol)} stosowany jest w sieciach Ethernet z IPv4, był też używany w sieciach Token Ring.
    W wersji IPv6 protokół ARP nie jest w ogóle wykorzystywany, zastępują go inne mechanizmy.

    \textbf{Ramka ARP}
    \begin{itemize}
        \item \textbf{Typ sprzętu} (2 oktety)
        \item \textbf{Typ protokołu} (2 oktety)
        \item \textbf{Długość adresu sprzętu} (1 oktet)
        \item \textbf{Długość adresu protokołu} (1 oktet)
        \item \textbf{Kod operacji} (2 oktety)
        \item \textbf{Adres sprzętu nadawcy} (dla Ethernet 6 oktetów)
        \item \textbf{Adres protokołu nadawcy} (dla IPv4 4 oktety)
    \end{itemize}

    \subsubsection{Wykrywanie zduplikowanych adresów IP}
    Tzw \textbf{"zbędny ARP"}
    \begin{itemize}
        \item Węzeł wysyła ARP Request z \textbf{zapytaniem o swój własny adres}.
        \begin{itemize}
            \item Jeśli ARP Reply \textbf{nie nadejdzie} to znaczy, że w lokalnym segmencie \textbf{nie ma konfliktu} adresów.
            \item Jeśli odpowiedź nadejdzie, oznacza to konflikt.
        \end{itemize}
        \item Węzeł już skonfigurowany traktowany jest jako węzeł z poprawnym adresem (\textbf{węzeł zgodny}), węzeł
        wysyłający „zbędny ARP” jest \textbf{węzłem konfliktowym}.
        \item \textbf{Węzeł konfliktowy wprowadza błąd} w pamięci podręcznej ARP komputerów w \textbf{całym segmencie} sieci. ARP Reply z węzła zgodnego nie naprawia sytuacji (ramka ARP Reply nie jest ramką rozgłoszeniową).
    \end{itemize}

    \subsubsection{Proxy ARP}
    Router ze skonfigurowanym mechanizmem Proxy ARP \textbf{odpowiada na ramki ARP Request w
    imieniu wszystkich} węzłów – komputerów spoza segmentu sieci lokalnej. Może
    być używany jest np. w sytuacji, gdy komputery w sieci nie mają ustawionego domyślnego
    routera. Routery mogą mieć włączoną standardowo
    opcję Proxy ARP, wówczas jeśli jakiś komputer wyśle ARP Request z adresem spoza danej
    sieci lokalnej (zwykle to nie następuje), to router odpowie „w imieniu” komputera
    zewnętrznego.


    \subsubsection{Komunikacja między komputerami}
    Komputer źródłowy K1 (IP1, MAC1), docelowy K2 (IP2, MAC2, WW2).


    Jeżeli na K1 ktoś spróbuje otworzyć WWW2, to:

    \begin{itemize}
        \item Zadziała system DNS: K1 skontaktuje się ze swoim serwerem DNS i zapyta jaki jest adres IP komputera związanego z nazwą domenową WW2. Serwer DNS znajdzie odpowiedni adres w swoich zasobach i odeśle informację do K1.
        \item Przeglądarka utworzy komunikat (wg protokołu HTTP). Do komunikatu zostanie dodany nagłówek (wg protokołu TCP), który zawiera m.in. port docelowy (standardowo 80) oraz port źródłowy. Komunikat razem z dołączonym nagłówkiem TCP nazywa się \textbf{segmentem TCP}.
        \item Do segmentu TCP zostanie dodany nagłówek IP – w ten sposób powstanie \textbf{pakiet IP}.
        \item Pakiet musi być przesłany w ramce. Do pakietu musi zostać dodany nagłówek ramki, zawierający źródłowy i docelowy adres MAC. \textbf{K1 nie zna adresu MAC2}. Zna tylko IP2. Wykorzystywany jest \textbf{protokół ARP}.
        \begin{itemize}
            \item K1 wysyła \textbf{ARP Request} (ta NIE zawiera pakietu IP), która ma adres rozgłoszeniowy jako adres docelowy.
            \item Każdy komputer przyłączony do przełącznika ma obowiązek odebrać ramkę wysłaną na adres rozgłoszeniowy MAC. Jednak tylko komputer o zadanym IP odpowie na ARP Request.
            \item Odpowiedź \textbf{ARP Reply} jest wysyłana na adres MAC komputera 1.
        \end{itemize}
        \item Po tym, jak K1 pozna adres MAC2 może już zbudować ramkę przeznaczoną do K2. Ramka jest wysyłana do przełącznika, a przełącznik dostarcza ją tylko do K2.
        \item K2 odbiera ramkę, sprawdza adres MAC docelowy i sumę kontrolną, po czym „wyjmuje” z ramki pakiet IP. Sprawdza adres docelowy IP i „wyjmuje” z pakietu segment TCP. Sprawdza do którego portu należy przekazać zawartość (komunikat HTTP) i ostatecznie „wyjmuje” komunikat http z segmentu i przekazuje do portu 80, na którym nasłuchuje serwer WWW.
        \item Serwer WWW konstruuje odpowiedź – stronę WWW. Strona ta zostanie umieszczona w komunikacie http, który następnie musi być przesłany do K1. Mechanizm jest analogiczny jak poprzednio.
    \end{itemize}

    \begin{tabular}{|c|c|c|c|c|}
        \hline
        \textbf{Nagłówek ramki }& \textbf{Nagłówek IP }& \textbf{Nagłówek TCP }& \textbf{Komunikat HTTP }& \textbf{Suma kontrolna}\\
        (numery MAC) & (numery IP) & (numery portów) & & \\
        & 20 bajtów & 20 bajtów & & 4 bajty\\
        \hline
    \end{tabular}

    \textbf{W przypadku komunikacji między komputerami rozdzielonymi przynajmniej jednym routerem} ramka wysyłana jest do bramy domyślnej (ARP na bramę), gdzie jest niszczona
    i nowa jest przekazywana dalej wg tego co w nagłówku pakietu IP.



\end{document}