\documentclass[../main.tex]{subfiles}
\begin{document}

    \textbf{Routowanie} - proces przesyłania pakietów (datagramów IP) od hosta nadawczego do
    odbiorcy na ogół z wykorzystaniem routerów pośredniczących. Każdy host oraz router podejmuje decyzję jak przesłać datagram, podejmowaną na podstawie tzw. tabel routowania oraz pewnych reguł.
    \textbf{Routing statyczny} - gdy w routerze tabela routowania jest wypełniona wpisami statycznymi i nie zmienia się.\\
    \textbf{Routing dynamiczny} - tabela routowania zmienia się dynamicznie na podstawie protokołów routowania.\\
    Protokoły routowalne: \textbf{IPv4}, IPX firmy Novell (należący do stosu IPX/SPX), AppleTalk, IPv6. Routery mogą realizować trasowanie dla pakietów z
    protokołów innych niż IPv4.\\
    Dla routerów tablica routowania na ogół jest modyfikowana dynamicznie na podstawie \textbf{protokołów routowania}. Określają one w jaki sposób routery mają wymieniać między sobą informacje na temat połączeń między routerami w sieci i jak na podstawie tych informacji mają aktualizować swoje tablice routowania.\\
    Protokoły routowania: RIP, RIP2, OSPF, IGRP, EIGRP, BGP.
    \textbf{Brama domyślna} to router, do którego kierowany jest datagram, jeśli nie została znaleziona dla niego lepsza trasa w tablicy routowania.\\

    Typy bezpośrednich połączeń:
    \begin{itemize}
        \item rozgłoszenia - np. Ethernet; działa protokół ARP,
        \item punkt-punkt - np. analogowa linia telefoniczna; technologie sieci rozległych WAN, ARP nie jest wykorzystywany;
        \item wielodostęp nierozgłoszeniowy (NBMA) -  technologie przełączania pakietów WAN Frame Relay, ATM; ARP nie jest wykorzystywany.
    \end{itemize}

    \subsubsection{Tablica routowania IP}
    Zawiera wpisy dotyczące tras do hostów, routerów i sieci. Zwykle w
    hostach liczba wpisów jest dużo mniejsza niż w routerach i zawiera informacje o bramie domyślnej.\\

    W każdym hoście i routerze dla każdego przesyłanego pakietu IP na podstawie tablicy
    routowania wyznaczane są dwie wartości:
    \begin{itemize}
        \item interfejs – reprezentacja fizycznego urządzenia, przez które ma być wysłany pakiet,
        \item adres następnego skoku - adres IP następnego routera, do którego ma być skierowany datagram lub adres
        docelowy hosta, jeśli jest bezpośrednio dołączony do nadawcy\
\end{itemize}

    Tablica routowania zawiera następnujące pola:
    \begin{itemize}
        \item \textbf{Przeznaczenie (Destination)}\\
        W koniunkcji (logiczne AND) z polem Maska zawiera informację o zakresach adresów IP, które są dostępne przy użyciu tej trasy. Pole to może zawierać ID
        sieci lub adres IP konkretnego hosta lub routera. W prawidłowym wpisie nie może zawierać jedynek w miejscu, gdzie w Masce sieci są zera.
        \item \textbf{Maska sieci}\\
        Zawiera maskę sieci, może też zawierać same jedynki (255.255.255.255).
        \item \textbf{Adres następnego skoku}\\
        Adres IP, do którego datagram będzie przesłany w następnym kroku, jeśli zostanie wybrana ta trasa. Bez znaczenia dla połączeń punkt-punkt.
        \item \textbf{Interfejs}\\
        Oznaczenie interfejsu, przez który datagram będzie przesłany do miejsca określonego przez adres następnego skoku.
        \item \textbf{Metryka}\\
        Liczba wskazująca na koszt trasy. Im wyższa wartość, tym „gorsza” trasa.
        \item \textbf{Odległość administracyjna}\\
        Liczba przypisywana na podstawie tego, w jaki sposób router poznał trasę do danego miejsca docelowego. „Wygrywa” trasa z najmniejszą liczbą. Trasa domyślna może być oznaczana przez 0.0.0.0 w polu Przeznaczenie i 0.0.0.0 w polu Maska sieci.
    \end{itemize}

    Proces określania trasy na podstawie tablicy routowania:
    \begin{itemize}
        \item Dla każdej trasy w tablicy routowania określa się, czy jest ona \textbf{zgodna z adresem IP} w przesyłanym pakiecie. Trasa domyślna jest traktowana zawsze jako zgodna.
        \item Spośród tras zgodnych wybierana jest ta (lub kilka), dla której w polu \textbf{Maska sieci} jest \textbf{największa liczba jedynek}. Może się zdarzyć, że jedyną trasą zgodną jest trasa domyślna.
        \item Spośród tras, które zostały wybrane w punkcie 2 wybierane są trasy o \textbf{najmniejszej metryce}.
        \item Spośród tras wybranych w punkcie 3 wybierana jest dowolna trasa.
    \end{itemize}
    Routery potrafią również wykonywać równoważenie obciążeń.

    \subsubsection{Komunikaty ICMP o przekierowaniu}
    Komunikaty ICMP o przekierowaniu pozwalają hostom TCP/IP na konfigurację tylko jednego
    routera – bramy domyślnej nawet w sytuacji, gdy w sieci lokalnej są dwa lub więcej
    routerów, które są odpowiedzialne za pewne miejsca docelowe.
    Hosty mogą zacząć pracę z jedną domyślną trasą i uczyć się topologii sieci (w szczególności
    informacji o routowaniu) poprzez otrzymywanie komunikatów ICMP.\\

    Komunikaty ICMP o przekierowaniu powinny być generowane przez routery, ale korzystać z
    nich mogą tylko hosty.
    Jeśli datagram IP zostanie celowo usuwanięty przez router, to może być wysłany odpowiedni komunikat protokołu ICMP do nadawcy.




\end{document}